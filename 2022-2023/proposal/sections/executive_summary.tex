%%%%%%%%%%%%%%%%%%%%%%%%%%%%%%%%%%%%%%%%%%%%%%%%%%%%%%%%%%%%%%%%%%%%%%%%%%%%%%%%%%%%%%%%%
%%%%%%%%%%%%%%%%%%%%%%%%%%%%         EXECUTIVE SUMMARY         %%%%%%%%%%%%%%%%%%%%%%%%%%
%%%%%%%%%%%%%%%%%%%%%%%%%%%%%%%%%%%%%%%%%%%%%%%%%%%%%%%%%%%%%%%%%%%%%%%%%%%%%%%%%%%%%%%%%

\section{Executive Summary} % (10 Points)
\label{sec:ExecutiveSummary}
% Section Requirements:
% 1) Objective Statement
% 2) Planned approach to achieve all objectives
% 3) \textbf{Includes main points from subsequent sections} (we lost points on this in 2020)

This document outlines and proposes the design-build-fly (DBF) process to be implemented by the Brigham Young University (BYU) Team in the AIAA 2022-2023 DBF competition.  

Our approach, is to iteratively design, build, and fly... [describe aircraft capabilities]

[consider trimming down or removing the rest of this.]
This proposal is organized as follows:  We first describe team management, organizational structure, schedule, and budget for this year in \cref{sec:ManagementSummary}.  We then describe our conceptual design approach in \cref{sec:ConceptualDesign} beginning with a decomposition of the mission requirements followed by a sensitivity study, written description of our design concept, and select images visually showing our aircraft concept.  In \cref{sec:ManufacturingPlan} we describe the various phases in our build flow shown in \cref{fig:manufacturingplan}, including specific materials and methods critical to bringing our design into reality. We finish in \cref{sec:TestingPlan} with our plans for phased ground and flight testing, including brief mentions of those we have already completed.
