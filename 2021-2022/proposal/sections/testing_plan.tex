%%%%%%%%%%%%%%%%%%%%%%%%%%%%%%%%%%%%%%%%%%%%%%%%%%%%%%%%%%%%%%%%%%%%%%%%%%%%%%%%%%%%%%%%%
%%%%%%%%%%%%%%%%%%%%%%%%%%%%%%          TESTING PLAN         %%%%%%%%%%%%%%%%%%%%%%%%%%%%
%%%%%%%%%%%%%%%%%%%%%%%%%%%%%%%%%%%%%%%%%%%%%%%%%%%%%%%%%%%%%%%%%%%%%%%%%%%%%%%%%%%%%%%%%

\section{Testing Plan} % (15 points)
\label{sec:TestingPlan}
% Section Requirements:
% 1) Component and ground test plan
% 2) Flight test plan 

As mentioned in \cref{ssec:ManufacturingFlow} and shown in \cref{fig:plannedtiming}, each of our design and build iterations culminate in testing.  Testing is divided into two categories as follows:
{\color{BYUred}[NEED TO FLESH OUT DETAILS BELOW BASED ON THE SPECIFICS OF THE COMPETITION AND YOUR CONCEPTUAL DESIGN.]}


%%%%%%%%%%%%%%%%%%%%%%%%%%%%
%%%% - Ground Testing - %%%%
%%%%%%%%%%%%%%%%%%%%%%%%%%%%
\subsection{Component/Ground Test Plan}
\label{ssec:GroundTestingPlan}

For all phases, ground testing will start roughly a week after prototyping has commenced. 

\subsubsection{Phase 1} We began by testing a quick series of concept prototypes for our {\color{BYUred}[PAYLOAD, WING FOLDING MECHANISM, LAUNCH STATION, OR WHATEVER THEY ARE THIS YEAR]} in order to quickly narrow down our brainstorming to the most viable solutions.

\subsubsection{Phase 2} In our preliminary testing phase, we will be looking at functioning prototypes of {\color{BYUred}[PAYLOAD, WING FOLDING MECHANISM, LAUNCH STATION, OR WHATEVER THEY ARE THIS YEAR]} in order to nail down the major details of the design.  This will prepare us for integration in the next phase.  In this phase, we will also be begin performing preliminary wind tunnel testing to validate our propulsion system. In addition, we will perform preliminary structural testing of our anticipated wing and other critical structures.

\subsubsection{Phase 3} Finally, we will integrate all the components and do dry runs of the ground mission, as well as final wind tunnel and structural testing to validate our detailed computational analyses.





%%%%%%%%%%%%%%%%%%%%%%%%%%%%
%%%% - Flight Testing - %%%%
%%%%%%%%%%%%%%%%%%%%%%%%%%%%
\subsection{Flight Test Plan}
\label{ssec:FlightTestingPlan}

In all phases, flight tests will typically take place at the end of the phase, in the week following the termination of the prototyping.

\subsubsection{Phase 1} Flight testing began with our concept prototype: a hand-launched, unpowered, uncontrolled glider.  Our primary goals for the concept test were to validate our static stability and general structural calculations, as well as illuminate any gotchas we may have missed in our initial design phase.

\subsubsection{Phase 2} Our preliminary flight test prototype will be a powered, controlled aircraft, though without the full competition functionality.  Our goal for the preliminary test is to validate our preliminary designs before moving on to detailed design aspects and full system integration, as well as note any unexpected behavior in the aircraft dynamic responses. 

\subsubsection{Phase 3} Our detailed design prototype will be complete enough that if desired, we could compete without building another iteration. Our goal for the final testing phase will be to fly the complete mission sequence, allowing for any final fine-tuning of the design before building our competition aircraft.
