%%%%%%%%%%%%%%%%%%%%%%%%%%%%%%%%%%%%%%%%%%%%%%%%%%%%%%%%%%%%%%%%%%%%%%%%%%%%%%%%%%%%%%%%%
%%%%%%%%%%%%%%%%%%%%%%%%%%%%         EXECUTIVE SUMMARY         %%%%%%%%%%%%%%%%%%%%%%%%%%
%%%%%%%%%%%%%%%%%%%%%%%%%%%%%%%%%%%%%%%%%%%%%%%%%%%%%%%%%%%%%%%%%%%%%%%%%%%%%%%%%%%%%%%%%

\section{Executive Summary} % (10 Points)
\label{sec:ExecutiveSummary}
% Section Requirements:
% 1) Objective Statement
% 2) Planned approach to achieve all objectives
% 3) \textbf{Includes main points from subsequent sections} (we lost points on this in 2020)

This document outlines and proposes the design-build-fly (DBF) process to be implemented by the Brigham Young University (BYU) Team in the AIAA 2021-2022 DBF competition.  
% This year's design will be a remotely controlled aircraft that can efficiently fly a designated route while carrying complicated payloads which include syringes, and shock sensitive vial packages.  The aircraft will also need to be able to automatically unload the vial packages one at a time, without inducing more than a 5g acceleration on the boxes, while taking off to fly another lap in between each drop sequence.  This is to simulate delivery of fragile vaccine vials to multiple locations.
Our approach, is to iteratively design, build, and fly a single wing, dual rotor, large capacity aircraft with semi-autonomous payload deposition capabilities. 
The design we propose is suitable for the competitive completion of the ground mission (GM) as well as the 3 flight missions (FM). The GM requires the rapid loading and unloading of the various payload units, necessitating the intuitive and accessible design described below. The flight mission requirements are as follows: FM1 requires the completion of 3 laps around the flight course within 5 minutes, carrying payload management hardware without any payload packages. The FM2 requirements are to carry as many 30 mL syringes as possible, as quickly as possible, within the time, energy, and weight constraints\footnote{See DBF 2022 Rules for weight and energy constraints, \url{https://www.aiaa.org/docs/default-source/uploadedfiles/aiaadbf/dbf-rules-2022.pdf}} for the competition. In an effort to maximize our FM2 score, we have selected a large payload capacity design. In addition, a dual rotor configuration increases the efficacy of the propulsion system in flight (avoiding obstruction from the fuselage), on-ground agility (differential thrust), and decreases takeoff distance (induction of wing lift). The requirements for FM3 are to transport and individually drop off as many mock vial packages as possible (though no more than one tenth the number of syringes carried in FM2) within the allowed time limits, all without exceeding 5 g accelerations. Our passive and active suspension systems will allow the gentle completion of the FM3 objectives.
% Our approach to this problem, as outlined in detail in the sections of this proposal, will be to design an aircraft that maximizes the number of payload units we can carry and safely unload in ten minutes.  Members of our team who have high levels of experience with aircraft design, and have experience with this competition will be valuable contributors to our efforts. Limitations to consider include a maximum battery power of one hundred watts, a maximum wingspan/fuselage length of eight feet, and a mission time of under ten minutes.  

This proposal is organized as follows:  We first describe team management, organizational structure, schedule, and budget for this year in \cref{sec:ManagementSummary}.  We then describe our conceptual design approach in \cref{sec:ConceptualDesign} beginning with a decomposition of the mission requirements followed by a sensitivity study, written description of our design concept, and select images visually showing our aircraft concept.  In \cref{sec:ManufacturingPlan} we describe the various phases in our build flow shown in \cref{fig:manufacturingplan}, including specific materials and methods critical to bringing our design into reality. We finish in \cref{sec:TestingPlan} with our plans for phased ground and flight testing, including brief mentions of those we have already completed.
